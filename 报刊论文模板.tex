%!TEX program = xelatex

\documentclass[a4pper]{article}
\usepackage[UTF8]{ctex}
\usepackage{geometry}
    \geometry{left=2.0cm,right=2.0cm,top=2.3cm,bottom=2.6cm}
\usepackage{siunitx}
\usepackage{amsmath}
\usepackage{setspace}
    \onehalfspacing
\usepackage{listings}
\usepackage{xcolor}
\usepackage{multicol}%包
% \setlength{\columnseprule}{0.5pt}
% \def\columnseprulecolor{\color{black}}%定义分割线的颜色
\usepackage{indentfirst} % 中文首段缩进
\renewcommand{\baselinestretch}{1.3} %定义行间距

\usepackage{titlesec}   %设置页眉页脚的宏包

\title{应用密码学\\ PGP 协议的应用和安全性分析}
\author{@wwyqianqian}
\date{} %清除日期


\begin{document}
\maketitle          %添加这一句才能够显示标题等信息
\tableofcontents    %添加目录
\clearpage          %转下一页


\newpagestyle{main}{            
    \sethead{Central China Normal University}{School of CS}{Dec 2019}     %设置页眉

    \headrule             %添加页眉的下划线
    \footrule             %添加页脚的下划线
}
\pagestyle{main}    %使用该style

\begin{abstract}
该部分内容是放置摘要信息的。该部分内容是放置摘要信息的。该部分内容是放置摘要信息的。该部分内容是放置摘要信息的。该部分内容是放置摘要信息的。\\
{\bf Keywords:} keywords1, keywords2, keywords3, keywords4, keywords5,keywords6
\end{abstract}



\begin{multicols}{2}
\section{section1}
    \subsection{subsection1}
    内容1,这是一个双栏显示。\cite{ref1}内容1,这是一个双栏显示。内容1,这是一个双栏显示。内容1,这是一个双栏显示。内容1,这是一个双栏显示。内容1,这是一个双栏显示。内容1,这是一个双栏显示。内容1,这是一个双栏显示。内容1,这是一个双栏显示。内容1,这是一个双栏显示。内容1,这是一个双栏显示。内容1,这是一个双栏显示。\cite{ref1, ref4}


\section{section2}
    \subsection{subsection1}
    内容2.1,这是一个双栏显示。内容2.1,这是一个双栏显示。内容2.1,这是一个双栏显示。内容2.1,这是一个双栏显示。内容2.1,这是一个双栏显示。内容2.1,这是一个双栏显示。
    \subsection{subsection2}
    内容2.2,这是一个双栏显示。内容2.2,这是一个双栏显示。内容2.2,这是一个双栏显示。内容2.2,这是一个双栏显示。内容2.2,这是一个双栏显示。内容2.2,这是一个双栏显示。内容2.2,这是一个双栏显示。
    \subsection{subsection3}
    内容2.3,这是一个双栏显示。内容2.3,这是一个双栏显示。内容2.3,这是一个双栏显示。内容2.3,这是一个双栏显示。内容2.3,这是一个双栏显示。

\section{section3}
    内容3,这是一个双栏显示。内容3,这是一个双栏显示。内容3,这是一个双栏显示。内容3,这是一个双栏显示。内容3,这是一个双栏显示。

\section{section4}
    \subsection{subsection1}
    内容4.1,这是一个双栏显示。内容4.1,这是一个双栏显示。内容4.1,这是一个双栏显示。内容4.1,这是一个双栏显示。内容4.1,这是一个双栏显示。内容4.1,这是一个双栏显示。
    \subsection{subsection2}
    内容4.2,这是一个双栏显示。内容4.2,这是一个双栏显示。内容4.2,这是一个双栏显示。内容4.2,这是一个双栏显示。内容4.2,这是一个双栏显示。内容4.2,这是一个双栏显示。内容4.2,这是一个双栏显示。内容4.2,这是一个双栏显示。内容4.2,这是一个双栏显示。内容4.2,这是一个双栏显示。内容4.2,这是一个双栏显示。内容4.2,这是一个双栏显示。内容4.2,这是一个双栏显示。
    \subsection{subsection3}
    内容4.3,这是一个双栏显示。内容4.3,这是一个双栏显示。内容4.3,这是一个双栏显示。内容4.3,这是一个双栏显示。内容4.3,这是一个双栏显示。内容4.3,这是一个双栏显示。
    \subsection{subsection4}
    内容4.4,这是一个双栏显示。内容4.4,这是一个双栏显示。内容4.4,这是一个双栏显示。内容4.4,这是一个双栏显示。内容4.4,这是一个双栏显示。内容4.4,这是一个双栏显示。内容4.4,这是一个双栏显示。内容4.4,这是一个双栏显示。内容4.4,这是一个双栏显示。内容4.4,这是一个双栏显示。内容4.4,这是一个双栏显示。内容4.4,这是一个双栏显示。内容4.4,这是一个双栏显示。内容4.4,这是一个双栏显示。

\section{section5}
    内容5,这是一个双栏显示。内容5,这是一个双栏显示。内容5,这是一个双栏显示。内容5,这是一个双栏显示。内容5,这是一个双栏显示。内容5,这是一个双栏显示。

\section{section6}
    内容6,这是一个双栏显示。内容6,这是一个双栏显示。内容6,这是一个双栏显示。内容6,这是一个双栏显示。内容6,这是一个双栏显示。内容6,这是一个双栏显示。内容6,这是一个双栏显示。内容6,这是一个双栏显示。内容6,这是一个双栏显示。
    \subsection{subsection1}
    \subsection{subsection2}
    \subsection{subsection3}


\begin{thebibliography}{40}
\bibitem{ref1}Zheng L, Wang S, Tian L, et al., Query-adaptive late fusion for image search and person re-identification, Proceedings of the IEEE Conference on Computer Vision and Pattern Recognition, 2015: 1741-1750.
\bibitem{ref2}冯西桥.核反应堆压力管道与压力容器的LBB分析[R].北京:清华大学核能技术设计研究院,1997:9-10.
\bibitem{ref3}刘伟.汉字不同视觉识别方式的理论和实证研究[D].北京:北京师范大学心理系,1998.
\bibitem{ref4}中华人民共和国科学技术委员会.科学技术期刊管理办法[Z].1991—06—05




\end{thebibliography}

\end{multicols}
\end{document}