%!TEX program = xelatex

\documentclass{article}
\usepackage[UTF8]{ctex}
\usepackage{geometry} 
    \geometry{left=2.0cm,right=2.0cm,top=3.0cm,bottom=3.0cm}

\usepackage{siunitx}
\usepackage{amsmath}

\usepackage{setspace}
    \onehalfspacing
 

\begin{document}
    \vspace*{2cm}
    \begin{minipage}{\textwidth}
    	\begin{minipage}[t]{0.48\textwidth}
    	\centering
    	\makeatletter\def\@captype{table}
    	    \makeatother
      		\begin{tabular}{|c|c|} 
   			\hline
 			姓名 & 千千w \\
 			\hline
 			学号 & 1919810 \\
 			\hline
    		\end{tabular}
    	\end{minipage}
    	\begin{minipage}[t]{0.48\textwidth}
    	\centering
    	\makeatletter\def\@captype{table}
    	    \makeatother
            \begin{tabular}{|c|c|}
 			\hline
 			实验成绩 & \ \ \ \ \ \ \ \ \ \\
 			\hline
 			\end{tabular}
  		\end{minipage}
    \end{minipage}
    \vspace*{2cm}


	\begin{center}
		\Huge{\textbf{华中师范大学计算机科学与技术系}}

		\Huge{\textbf{实验报告书}}
	\end{center}
	
	\vspace*{3cm}
	\begin{center}
	\begin{Large}
		\begin{tabular}{p{3cm} p{7cm}<{\centering}}
			实验题目: & 3.7 - 串匹配问题 \\
			\hline
			课程名称: & 算法设计与分析 \\
			\hline
			主讲教师: & 张茂元 \\
			\hline
			辅导教师: & 张茂元 \\
			\hline
			课程编号: & 48710003 \\
			\hline
			班\qquad 级: & 周一周三第 3-4 节课堂 \\
			\hline
			实验时间: & 2019 年 3 月 6 日 \\
			\hline
		\end{tabular}
	\end{Large}
    \end{center}


\clearpage


\section{实验目的}
\section{实验内容}
\section{实验环境}
\section{实验设计原理}
\section{实验详细实现过程与算法流程}
\section{实验调试与结果分析(问题的发现、分析、解决方案与创新)}
\section{实验改进意见与建议}
\section{附录与说明}


\end{document}